\documentclass{ucph-handout}
\usepackage{wrapfig}
\newcounter{handout}
\newcommand{\Ark}{Ark \#\arabic{handout} -- }
%\renewcommand{\Title}{Arbejdsark \#\arabic{handout}}%
\renewcommand{\TimeAndLocation}{DIKU, 2019}%
\usepackage[bottom=3cm]{geometry}

\renewcommand{\Title}{\Ark Tegn med Processing.py}%
\renewcommand{\Author}{Martin Dybdal}
\renewcommand{\AuthorEmail}{dybber@di.ku.dk}

\begin{document}
\begin{exercisebox}[adjusted title=Første program]
Tast følgende eksempel ind i Processing editoren og tryk på
\includegraphics[height=4mm]{../illustrations/images/processing_play_button}-knappen:
\begin{python}
size(400, 400)

# Træ
rect(55, 50, 10, 20)
ellipse(60, 35, 30, 40)
\end{python}

\vspace{3mm}
\noindent
Gem med det samme projektet og kald det ``GreenCity''.

\vspace{3mm}
\noindent
Tilføj nu:
\begin{python}
# Kraftværk
rect(120, 50, 60, 30)
rect(160, 20, 10, 30)
triangle(120, 50, 136, 40, 136, 50)
triangle(136, 50, 152, 40, 152, 50)
\end{python}
\vspace{3mm}
\noindent
Og:
% \begin{python}
% # Person
% ellipse(270, 65, 8, 8)
% line(270, 70, 270, 75)
% line(270, 69, 277, 75)
% line(270, 69, 263, 75)
% line(270, 75, 275, 82)
% line(270, 75, 265, 82)
% \end{python}
\begin{python}
# Vindmølle
line(300, 50, 320, 51)
line(300, 50, 289, 67)
line(300, 50, 291, 32)
line(300, 50, 300, 90)
\end{python}

\tcbsubtitle{Opgave}
Prøv nu at tegne en bil og en sky:

\includegraphics[width=0.4\textwidth]{../illustrations/images/bil-streg.png}
\includegraphics[width=0.3\textwidth]{../illustrations/images/sky.png}
\end{exercisebox}

\begin{exercisebox}[adjusted title=Farver]
I skal nu farvelægge figurerne. Til det skal I bruge
\ttpy{fill(r, g, b)}-funktionen, der vælger hvilken farve der skal
bruges til udfyldning og tekstfarve. Det handler om at kalde
\ttpy{fill} de rigtige steder! Som argumenter angiver man mængden
af rød (0-255), blå (0-255) og grøn (0-255).

~

\noindent
Her er nogle grundfarver:

\begin{minipage}{0.45\linewidth}
\begin{python}
fill(255, 0, 0)  # rød
fill(0, 255, 0)  # grøn
fill(0, 0, 255)  # blå
\end{python}
\end{minipage}
\begin{minipage}{0.45\linewidth}
\begin{python}
fill(0, 0, 0)       # sort
fill(255, 255, 255) # hvid
fill(255, 255, 0)   # gul
\end{python}
\end{minipage}

Find eventuelt farver ved hjælp af en online farvevælger eller RGB
farve tabel. Søg for eksempel efter ``rgb color picker''.

\vspace{2mm}

\tcbsubtitle{Streger og omrids}
Til at angive farven på streger (fx \ttpy{line}) og omrids
bruges \ttpy{stroke(r, g, b)}. Prøv også funktionen \ttpy{noStroke()},
til at slå optegning af omrids fra.
\end{exercisebox}

\newpage
\begin{exercisebox}[adjusted title=Eksempel]
Her er et eksempel på hvordan det kan se ud efter farvelægning:
\vspace{1mm}
\begin{center}
\includegraphics[width=0.55\textwidth]{../illustrations/images/farvelagt.png}
\end{center}
\vspace{-1mm}
\end{exercisebox}

\begin{exercisebox}[adjusted title=Green City]
Brug det I har lært til at få det til at blive en lidt pænere scene,
med baggrund, forgrund og de ekstra detaljer, som I synes, der skal være
der. Senere skal vi udbygge projektet til en simulation, hvor bilen
skal oplades med strøm fra enten vindmølle eller kraftværket, men hvor
bilen helst skal lades op med grøn strøm fra vindmøllen.

Nedenfor er vist to eksempler, hvor der f.eks. er tegnet veje og
flyttet lidt rundt på figurerne. Der er ikke noget facit på, hvordan
det skal se ud.

\vspace{2mm}
\noindent
Tip: For at ændre baggrundsfarven fra grå  kan i bruge \ttpy{rect(0, 0, 400, 400)}, som i
allerede kender, men i kan også bruge kommandoen \ttpy{background(r, g, b)}.
Den sletter alt og udfylder skærmen med den angivne farve.

\begin{center}
  \includegraphics[width=0.49\textwidth]{../illustrations/images/elbil.png}
  \includegraphics[width=0.49\textwidth]{../illustrations/images/miniby.png}
\end{center}

\noindent
Husk at bruge kommentarer, så I nemt kan finde rundt i koden!
\end{exercisebox}
\newpage
\stepcounter{handout}
\renewcommand{\Title}{\Ark Variable}%
\begin{exercisebox}[adjusted title=Tegn en simpel fisk]
Opret et nyt projekt (``File'' -> ``New'') og gem med det samme
projektet. Kald det ``Akvarie''.

\noindent
Skriv denne stump kode ind:
\begin{python}
size(400, 400)
fishX = 150
ellipse(fishX, 200, 120, 75)
triangle(fishX - 60, 200, fishX - 90, 170, fishX - 90, 230)
\end{python}
Kør koden og prøv derefter at ændre 150 til et andet tal i angivelsen af \ttpy{fishX}.\\

\noindent
Tilføj nu følgende:
\begin{python}
eyeSize = 15
ellipse(fishX + 30, 190, eyeSize, eyeSize)
\end{python}
Prøv at ændre på værdien af \ttpy{eyeSize}.
\end{exercisebox}

\begin{exercisebox}[adjusted title=Opgaver]
I har nu tilføjet en fisk, der kan flyttes, bare ved at ændre én
værdi.

\begin{itemize}
\item Farvelæg fisken

\item Tegn en finne på siden af fisken vha. triangle()-kommandoen\footnote{Læs om \ttpy{triangle()} i dokumentationen: \url{https://py.processing.org/reference/triangle.html}}

\item Få finnen til at flytte med når I ændrer \ttpy{fishX}

\item Giv fisken en pupil, som flytter med når I ændrer \ttpy{fishX}
  
\item Lav en ny variabel, \ttpy{fishY}, der styrer fiskens y-position
\end{itemize}
\hspace{1cm}\includegraphics[width=0.4\textwidth]{../illustrations/images/fisk-fishY.png}

\noindent
Husk at gemme projektet. Vi skal arbejde videre med det senere.
\end{exercisebox}
\newpage
\begin{exercisebox}[adjusted title=Green City fortsat]
Skift over til Green City-projektet og indfør variable til angivelse af
objekternes placering, så vi senere kan animere objekterne.

\begin{itemize}
\item Lav en variabel \ttpy{carX}, så bilen kan bevæge sig frem
  og tilbage
\item Lav en variabel \ttpy{cloudX}, så skyen kan bevæge sig frem og tilbage
\item Lav en variabel \ttpy{treeX}, så træet kan flyttes horizontalt
\item Lav en variabel \ttpy{treeY}, så træet kan flyttes vertikalt
\end{itemize}

\begin{center}
\includegraphics[width=0.5\textwidth]{../illustrations/images/carX-cloudX-treeXY.png}
\end{center}
\end{exercisebox}

\begin{exercisebox}[adjusted title=Akvarie fortsat]
Brug nu det I har lært til at udvide akvarie projektet, her er et
eksempel, men brug gerne jeres fantasi! I eksemplet er der brugt en
variabel til x-koordinat af tangplanten, og en anden variabel til
x-koordinat af hele gruppen af sten (som en enhed). Der er oprettet et
ekstra sæt variable \ttpy{fish2X}/\ttpy{fish2Y} til at styre
placeringen af den ekstra fisk.
\begin{center}
\includegraphics[width=0.5\textwidth]{../illustrations/images/akvarie.png}
\end{center}
\end{exercisebox}


\newpage
\stepcounter{handout}
\renewcommand{\Title}{\Ark Funktioner}%

\begin{exercisebox}[adjusted title=Fiske-funktion]
Funktioner giver mulighed for at navngive hele blokke af kode,
genbruge den samme kode flere steder,
og sætte struktur på kode.\\

\noindent
Åbn akvarie-projektet. Tilføj følgende ``fiske-tegne-funktion''
nederst i projektet. BEMÆRK! Linjeindrykning med 4 mellemrum er vigtig!

\begin{python}
def drawSimpleFish(x, y):
    ellipse(x, y, 120, 75)
    triangle(x - 60, y, x - 90, y - 30, x - 90, y + 30)
\end{python}

\noindent
Derefter kan vi kalde funktionen således:
\begin{python}
drawSimpleFish(100,  50)
drawSimpleFish(300, 200)
drawSimpleFish(20, 20)
drawSimpleFish(80, 80)
\end{python}
Nu går det meget hurtigere med at få fyldt akvariet med fisk, og vi
undgår at kopiere kode.

\tcbsubtitle{Opgave}
Opret jeres egen \ttpy{drawFish(x, y)} funktion, der tegner hele jeres
fisk med farve, finner og øjne.
\end{exercisebox}

\begin{exercisebox}[adjusted title=Green City fortsat]
Ovre i Green City-projektet kan I også prøve at skrive en funktion
til at tegne træer:
\begin{python}
def drawTree(treeX):
    fill(100, 100, 0)
    rect(treeX - 5, 350, 10, 20)
    fill(0, 200, 0)
    ellipse(treeX, 335, 40, 50)

drawTree(160)
drawTree(300)
\end{python}

\noindent
Få sat struktur på koden til Green City-projektet ved hjælp af funktioner:
\begin{itemize}
\item Skriv en \ttpy{drawCloud(x)}-funktion, der tegner en sky
\item Udvid \ttpy{drawTree(x)}-funktionen til at også tage imod et y-koordinat
\item Skriv en \ttpy{drawCar(x)}-funktion, der tegner en bil
\item Skriv en \ttpy{drawPowerplant()}-funktion og en
  \ttpy{drawWindmill()}-funktion, der tegner hhv. kraftværket og
  vindmøllen. Vi får ikke behov for at flytte på dem, så de behøver
  ikke tage koordinater som argument.
\end{itemize}

\noindent
Kald alle funktionerne nederst i dit program. For eksempel:

\begin{python}
drawTree(150, 235)
drawTree(240, 335)
drawPowerplant()
drawWindmill()
drawCar(50)
drawCloud(280)
\end{python}
\end{exercisebox}
\newpage

~

\newpage
\stepcounter{handout}
\renewcommand{\Title}{\Ark Animation og funktionerne \ttpy{setup}/\ttpy{draw}}

\begin{exercisebox}[adjusted title=Simpel animation]
Opret et nyt midlertidigt projekt (I behøver ikke gemme det). Tast
denne stump kode ind:
\begin{python}
x = 50
  
def setup():
    size(400, 400)

def draw():
    global x
    background(255, 255, 255)
    fill(255, 0, 0)
    ellipse(x, 100, 30, 30)
    x = x + 1
\end{python}

\noindent
Funktionen \ttpy{draw} kaldes automatisk 60 gange i sekundet!

\tcbsubtitle{Opgaver}
\begin{itemize}
\item Prøv at ændre $50$ til et andet tal 
\item Prøv at ændre linjen \ttpy{x = x + 1} til \ttpy{x = x - 1} eller til \ttpy{x = x + 5}
\item Forsøg at flytte kaldet til \ttpy{background} fra \ttpy{draw}
  til \ttpy{setup} - hvad sker der?
\end{itemize}

\noindent
\tcbsubtitle{BEMÆRK!}
Når man bruger \ttpy{setup}/\ttpy{draw}, er det ikke
tilladt at \underline{\textit{kalde}} tegne-funktioner udenfor
\ttpy{setup} og \ttpy{draw}. Funktioner defineres udenfor setup og
draw, men alle \textit{kald} til tegnefunktioner skal flyttes ind i
enten \ttpy{setup} eller \ttpy{draw}.
\end{exercisebox}

\begin{exercisebox}[adjusted title=Akvarie fortsat]
\begin{itemize}
\item \emph{Omskrivning af akvarieprojektet til brug af \ttpy{setup}/\ttpy{draw}:}
  \begin{itemize}
  \item Tilføj tomme \ttpy{setup} og \ttpy{draw}-funktioner nederst i programmet
  \item Kald \ttpy{size(400, 400)} i \ttpy{setup}
  \item Kald alle tegnefunktionerne i \ttpy{draw}, inkl. tegning af baggrunden
  \end{itemize}
  
\item \emph{Få fiskene til at svømme:}
  \begin{itemize}
  \item Opret to globale variabler \ttpy{fish1X} og
    \ttpy{fish2X} (før \ttpy{setup}/\ttpy{draw})
  \item Brug de nye variable som x-argument, når I kalder \ttpy{drawFish()}
  \item \emph{HUSK} linjerne: ~\ttpy{global fish1X}~ og ~\ttpy{global fish2X}~
  \item Opdater variablerne med $+1$/$-1$ inde i \ttpy{draw}-funktionen
  \end{itemize}
\end{itemize}
\end{exercisebox}
\begin{exercisebox}[adjusted title=Green City fortsat]
\begin{itemize}
\item Opret to globale variabler \ttpy{carX} og \ttpy{cloudX}
\item Få bilen til at køre mod højre
\item Få skyen til at starte uden for billedet i højre side og bevæge sig mod venstre
\end{itemize}
\end{exercisebox}

\newpage
\begin{exercisebox}[adjusted title=Tilfældighed]
\begin{minipage}{0.65\linewidth}
Opret et helt nyt projekt, gem det som ``random\_circles''.
  
Tilføj følgende kode:
\begin{python}
def setup():
    size(400, 400)

def draw():
    x = random(0, width)
    y = random(0, height)
    ellipse(x, y, 30, 30)
\end{python}
\end{minipage}
\begin{minipage}{0.34\linewidth}
\begin{center}
\includegraphics[width=0.75\textwidth]{../illustrations/images/randomcircles.png}
\end{center}
\end{minipage}
\tcbsubtitle{Opgaver}
\vspace{-2mm}
\begin{itemize}
\item Få cirklerne til at ændre størrelse tilfældigt
\item Få cirklerne til at blive tegnet i tilfældige
  farver. Eksperimenter jer frem.
\end{itemize}
\vspace{-4mm}
\end{exercisebox}

\begin{exercisebox}[adjusted title=Input fra musen]
\vspace{-2mm}
Opret et nyt projekt og kald det ``Tegneprogram''. Til tegneprogrammet
skal vi bruge placeringen af musen, som kan aflæses via de indbyggede variable
\ttpy{mouseX} og \ttpy{mouseY}. Prøv først at indtaste følgende program:
\begin{python}
def setup():
    size(800, 800)
    background(255, 255, 255)
def draw():
    fill(0, 0, 0)
    ellipse(mouseX, mouseY, 5, 5)
  \end{python}
  \vspace{-4mm}
\end{exercisebox}

\begin{exercisebox}[adjusted title=Tastatur input]
\vspace{-2mm}
Tastetryk på tastaturet kan opdages ved at tilføje funktionen
\ttpy{keyPressed}\footnote{Dokumentation: \url{https://py.processing.org/reference/keyPressed.html}}. Prøv at tilføje følgende til tegneprogrammet:
\begin{python}
def keyPressed():
    background(255, 255, 255)
\end{python}
Start programmet og tryk på en vilkårlig tast på tastaturet. For at
tjekke efter en specifik tast kan variablen ``\ttpy{key}'' aflæses.
\begin{python}
def keyPressed():
    if key == 'c':
        background(255, 255, 255)
\end{python}
\vspace{-2mm}
\end{exercisebox}

\begin{exercisebox}[adjusted title=Kreativt tegneprogram]
\vspace{-2mm}
Opgaven er nu at lave et tegneprogram, der er mere kunstfærdigt. Tegn
fx linjer fra hjørner hen til musen. Husk at gemme projektet!
\begin{center}
  \includegraphics[height=3.5cm]{../illustrations/images/tegneprogram2.png}
  \quad
  \includegraphics[height=3.5cm]{../illustrations/images/tegneprogram.png}
\end{center}
\end{exercisebox}
\newpage
\stepcounter{handout}
\renewcommand{\Title}{\Ark Betingelser}%
\begin{exercisebox}[adjusted title=Eksperimentér]
Med betingelser kan vi få ting til at ske når specielle kriterier er
opfyldt. Åbn akvarie-projektet og prøv at indsætte følgende i
\ttpy{draw}-funktionen:
\begin{python}
if fish1X > 500:
    fish1X = -40
        
if fish1X < -50:
    fish1X = 450
\end{python}

\noindent
Hvad sker der?
\end{exercisebox}

\begin{exercisebox}[adjusted title=Skifte retning]
Ved at lave en variabel der indeholder retningen, som fisken svømmer,
kan vi få den til at ændre retning, når den når siderne.

\begin{itemize}
\item Definer en global variabel \ttpy{fish1XVelocity} og sæt den til $1$
\item Ændr \ttpy{fish1X = fish1X + 1} til \ttpy{fish1X = fish1X + fish1XVelocity}
\item Fjern den tidligere \ttpy{fish1X}-betingelse
\item Tilføj disse betingelser:
\begin{python}
if fish1X > 400:
    fish1XVelocity = -1
if fish1X < 0:
    fish1XVelocity = 1
\end{python}
\end{itemize}
\end{exercisebox}

\begin{exercisebox}[adjusted title=Ændre udseende]
Prøv at ændre \ttpy{drawFish}-funktionen, så den tager retningen af
fisken med som argument, og brug en betingelse til at tegne finner og
øjne forskelligt alt efter hvilken retning fisken svømmer.

\begin{minipage}{0.60\linewidth}
\begin{python}
def drawFish(x, y, eyeSize, velocity):
    ... tegn kroppen ...

    # Svømmer mod højre
    if velocity >= 0:
        ... tegn finner og øjne ...

    # Svømmer mod venstre
    if velocity < 0:
        ... tegn finner og øjne ...
\end{python}
\end{minipage}
\begin{minipage}{0.40\linewidth}
\includegraphics[width=0.70\textwidth]{../illustrations/images/fisk-begge-retninger.png}
\end{minipage}
\end{exercisebox}

\begin{exercisebox}[adjusted title=Gør akvarieprojektet færdigt]
Nu er vi sådan set færdig med akvarie-projektet. Tilføj eventuelt
flere elementer. F.eks. bobler der dukker op fra bunden og svømmer mod
overfladen.
\end{exercisebox}
\newpage


\begin{exercisebox}[adjusted title=Flere opgaver om betingelser]
Åbn Green City projektet og gør følgende:
\begin{itemize}
\item Få skyen til at flytte tilbage til start og komme forbi igen og
  igen, hver gang den rammer kanten
\item Få bilen til at vende om i begge ender.
\end{itemize}

\tcbsubtitle{Bilbatteri}
Bilen skal stoppe, når batteriet er tomt:
  \begin{itemize}
  \item Definer en global variabel \ttpy{carBattery} og sæt den til 100
  \item Reducer den med 0.1, hver gang \ttpy{draw} kaldes
  \item Vis batteriets status vha. \ttpy{text(string, x, y)}-funktionen.
    Husk at konvertere tallet til en string vha. \ttpy{str()}.\footnote{Læs om \ttpy{text()} i
      onlinedokumentationen
      \url{https://py.processing.org/reference/text.html}}
  \item Hvis batteriet er tomt (\ttpy{<= 0}), skal bilen stoppe (sæt velocity til 0)
  \end{itemize}

\noindent
Tilføj en \ttpy{keyPressed()}-funktion og gør så man kan oplade elbilen, når man trykker 'C':
\begin{itemize}
  \item Læg \ttpy{0.3} til \ttpy{carBattery}, hver gang der
    trykkes 'C' på tastaturet
  \item Hvis \ttpy{carBattery} overstiger 100, så sæt den til
    100, så man ikke kan lade til mere end 100\%
\end{itemize}


\tcbsubtitle{Skiftende vindhastighed}
Variablen \ttpy{frameCount} tæller, hvor mange gange \ttpy{draw} er
kørt, siden programmet startede. Det kan vi bruge til at ændre noget, fx
hver gang der er gået 300 frames (5 sekunder).

\noindent
Indtast dette i draw-funktionen i Green City-projektet:
\begin{python}
fill(0, 0, 0)
text(frameCount, 350, 20)
if frameCount % 300 == 0:
    print(frameCount)
\end{python}

\noindent
Ovenstående skal læses som: ``hver gang 300 går op i
\ttpy{frameCount}, udskriv \ttpy{frameCount}''. Operatoren \% hedder
modulus og beregner resten ved division, fx giver beregningen
\ttpy{5 % 2} resultatet 1 (2 går op i 5 to gange, og der er 1 til rest).
\\

\noindent
Det kan vi bruge til at opdatere vindhastigheden hvert sekund:
  \begin{itemize}
  \item Opret en global windSpeed variabel
  \item Opdater windSpeed med en ny tilfældig værdi hvert sekund
  \item Vis vindhastigheden vha. tegnekommandoen
    \ttpy{text()}
  \item Få skyen til at flytte sig efter vindhastigheden
  \end{itemize}
% \item
%   Tænd og sluk for kraftværket
%   \begin{itemize}
%   \item Opret en global variabel \ttpy{powerplantOn}
%   \item Gør så man kan tænde og slukke kraftværket med en tryk på 'p'
%   \item Tegn røgskyer over kraftværket når det er tændt
%   \item Opret en variabel \ttpy{co2emission} og læg lidt til så længe
%     kraftværket er tændt. Vis værdien med \ttpy{text()}
%   \end{itemize}
\end{exercisebox}

\newpage
\stepcounter{handout}
\renewcommand{\Title}{\Ark Tilstandsmaskiner (Finite state machines)}
\begin{exercisebox}
\begin{minipage}{1.0\linewidth}
\begin{wrapfigure}[9]{r}{0.15\textwidth}
    \vspace{-1em}
    \includegraphics[width=0.15\textwidth]{../illustrations/images/unitek_kombinationslaas.jpg}
\end{wrapfigure}
  
Som det første eksempel på en tilstandsmaskine skal vi lave en
elektronisk kombinationslås, som dem der bruges til adgangskontrol på
døre.

\noindent
Tilstandsdiagram:

\noindent
\includegraphics[width=0.85\textwidth]{../illustrations/graphviz/combinationLock_dumb}

Hent Processing.py koden til kombinationslåsen her:
\mbox{\url{http://kortlink.dk/ufdh}} og kopier den ind i et nyt Processing-projekt.

% Indtast følgende kode:
% \begin{python}
% lockState = "LOCKED"

% def setup():
%     size(400, 400)

% def draw():
%     background(255, 255, 255)
%     fill(0)
%     text(lockState, 20, 20)

% def keyPressed():
%     global lockState
%     if lockState == "LOCKED":
%         if key == '3':
%             lockState = "FIRST_CORRECT"
%         else:
%             lockState = "LOCKED"
%     elif lockState == "FIRST_CORRECT":
%         if key == '2':
%             lockState = "SECOND_CORRECT"
%         else:
%             lockState = "FIRST_CORRECT"
%     elif lockState == "SECOND_CORRECT":
%         if key == '6':
%             lockState = "UNLOCKED"
%         else:
%             lockState = "SECOND_CORRECT"
% \end{python}
\end{minipage}

\begin{itemize}
\item  Afprøv kombinationslåsen og følg med som tilstanden skifter. Hvis man
ikke kendte løsenet (passwordet), hvor mange forsøg kræver det så
at gætte sig frem?

\item Ændr koden, så løsenet i stedet er 524
\item Prøv at ændre koden til at følge dette diagram i stedet, hvor
  forkerte tryk sætter tilstanden tilbage til start:
  
\includegraphics[width=0.85\textwidth]{../illustrations/graphviz/combinationLock_resetting}

\item Tegn et udvidet tilstandsdiagram over kombinationslåsen, med en ekstra
  tilstand, så der kræves 4 cifre. Tilføj dernæst den ekstra tilstand i koden.
\end{itemize}
\end{exercisebox}
\begin{exercisebox}[adjusted title=Automatisk genlås efter 2 sekunder]
Lad os udvide låsen, så døren automatisk låser igen efter 2 sekunder,
hvilket svarer til 120 frames:

\noindent
\begin{center}
\includegraphics[width=1.0\textwidth]{../illustrations/graphviz/combinationLock_timeout}
\end{center}

\begin{itemize}
\item Opret en global variabel ``\ttpy{timer}'' og sæt den til 0
\item Sæt \ttpy{timer}-variablen til 120, så snart låsen bliver låst op, det vil
  sige, når den skifter \ttpy{lockState} til \ttpy{"UNLOCKED"}.
\item Tæl ned med timeren i hver frame (tilføj følgende til \ttpy{draw}-funktionen):
\begin{python}
global timer
if timer > 0:
    timer = timer - 1
\end{python}
\item Når timeren er talt helt ned, skal låsen åbnes. I
  \ttpy{draw}-funktionen skal I nu tjekke, om vi er i tilstanden
  \ttpy{"UNLOCKED"} og timeren samtidig er talt ned til 0. Indsæt
  følgende i \ttpy{draw}:
\begin{python}
if lockState == "UNLOCKED":
    if timer == 0:
        lockState = "LOCKED"
\end{python}
\end{itemize}
\end{exercisebox}

\newpage
\begin{exercisebox}[adjusted title=Sæt på pause]
Vi kan også lave en tilstandsmaskine til at styre det overordnede
niveau af et spil: Er spillet i gang? Er vi i menuen? Er spillet på
pause? Er vi game over?

Vi vil ikke gå så langt, men kun tilføje at man kan sætte
fiske-projektet på pause. Det vil sige følgende simple tilstandsmaskine:
\begin{center}
\includegraphics[width=0.5\textwidth]{../illustrations/graphviz/pause}
\end{center}

\noindent
Styringsmæssigt skal der laves følgende ændringer:
\begin{itemize}
\item Opret en ny global variabel \ttpy{main_state = "RUNNING"}
\item Tilføj en \ttpy{keyPressed()}-funktion:
  \begin{itemize}
  \item Hvis man er i \ttpy{"RUNNING"}-tilstanden og man trykker
    ``p'' skal tilstanden ændres til \ttpy{"PAUSED"}
  \item Hvis man er i \ttpy{"PAUSED"}-tilstanden og man trykker ``p''
    skal tilstanden ændres til \ttpy{"RUNNING"}
\end{itemize}
\end{itemize}

\noindent
Derudover, så skal hele \ttpy{draw()}-funktionen pakkes ind i en betingelse:
\begin{python}
def draw():
    if game_state == "RUNNING":
        # Kroppen af oprindelig draw-funktion flyttes herind
        # Både tegne og bevægelses-kommandoer
    elif game_state == "PAUSED":
        # Tegn pause-tekst
        background(61, 213, 255)
        textSize(32)
        fill(255, 255, 255)
        text("PAUSE", 150, 200)
\end{python}

\begin{center}
  \includegraphics[width=0.40\textwidth]{../illustrations/images/fisk-begge-retninger.png}
  \quad
\includegraphics[width=0.4\textwidth]{../illustrations/images/fish-paused}
\end{center}
\end{exercisebox}


% \chapter{Alarm}
% Et andet eksempel på en tilstandsmaskine er en tyverialarm. Den har
% tre tilstande ``DEAKTIVERET'', ``AKTIV'' og ``ALARM''. Her er
% tilstandsdiagrammet:

% \noindent
% \includegraphics[width=1.0\textwidth]{../illustrations/graphviz/alarm}

% \noindent
% Opret et nyt projekt og indtast koden herunder. Den indeholder
% funktionalitet til at opdage et indbrud, ``bevægelse registreret'' i
% figuren ovenfor. Det sker når man bevæger musen ind i cirklen i
% midten.
% \begin{python}
% alarmState = "DEAKTIVERET"

% def setup():
%     size(400, 400)

% def draw():
%     global alarmState
%     background(255, 255, 255)
%     fill(0)
%     text(alarmState, 20, 20)
%     text("Slaa alarmen til/fra ved at trykke 's'", 20, 40)

%     noFill()
%     ellipse(200, 200, 100, 100)

%     if alarmState == "AKTIV":
%         # Tjek om musens cursor er inde i cirklen
%         if dist(mouseX, mouseY, 200, 200) < 50:
%             alarmState = "ALARM"
%     elif alarmState == "ALARM":
%         text("Indbrud", 20, 60)

% def keyPressed():
%     # indtast kode til at slå alarmen til/fra her
% \end{python}

% \noindent
% Det er nu din opgave at tilføje de sidste dele der gør det muligt at
% slå alarm til/fra og at alarmen stopper automatisk efter en timeout.
% \begin{itemize}
% \item \emph{Slå til/fra:} Tilføj en \ttpy{keyPressed} funktion der tillader at slå alarmen
%   til og fra når der trykkes 's'
% % \item Ændr koden i \ttpy{draw()} så alarmens tilstand sættes til ALARM, når musen er
% %   inde i cirklen OG tilstanden er AKTIV, i stedet for at der bare
% %   skrives ``Indbrud'' på skærmen.
% \item \emph{Time out:} Tilføj en timer, der sættes til 180 (3
%   sekunder) når alarmen går i gang, og som slår alarmen fra igen når
%   den er nede på 0 (dvs. sætter alarmState til \ttpy{"AKTIV"}).
% \end{itemize}


\newpage
\stepcounter{handout}
\renewcommand{\Title}{\Ark Tilstandsmaskiner (Finite state machines)}
\begin{exercisebox}[adjusted title=Tilstand for bilen]
Vi skal nu bruge en tilstandsmaskine til at få afsluttet
Green City-projektet, men det kommer til at tage nogle skridt. Først skal vi
have omskrevet logikken i bilen til at bruge en tilstandsmaskine i
stedet for ``carXVelocity''. Vi har lige nu 3 tilstande:
\begin{itemize}
\item \ttpy{"FORWARD"} - bilen kører mod højre
\item \ttpy{"BACKWARD"} - bilen kører mod venstre
\item \ttpy{"OUT_OF_FUEL"} - bilen er løbet tør for strøm og spillet er slut
\end{itemize}
\begin{center}
\includegraphics[width=\textwidth]{../illustrations/graphviz/carStateMachine_simple}
\end{center}

\begin{itemize}
\item Når vi er i FORWARD tilstanden, flyttes bilen fremad, batteriet
  mister strøm
\item Når vi er i BACKWARD tilstanden, flyttes bilen bagud, batteriet mister strøm
\item Når vi er i OUT\_OF\_FUEL tilstanden, skal der ikke ske noget. Spillet er slut.
\end{itemize}

\tcbsubtitle{Skriv det om til kode}
Koden skal have følgende struktur. Din opgave er at udfylde ``...''
med koden der flytter bilen, skifter tilstand og bruger strøm fra
batteriet.
\begin{python}
    if carState == "FORWARD":
        ...
        if carX > 400:
            ...
        if carBattery <= 0:
            ...
    elif carState == "BACKWARD":
        ...
        if carX < 0:
            ...
        if carBattery <= 0:
            ...
    elif carState == "OUT_OF_FUEL":
        pass # pass means "do nothing"
\end{python}
\end{exercisebox}
\newpage
\begin{exercisebox}[adjusted title=Udvidet tilstandsmaskine]
Vi skal nu have gjort, så bilen parkerer, hver gang den når venstre side
af skærmen. Derfor udvider vi vores tilstandsmaskine med en
parkeringstilstand:
\begin{center}
\includegraphics[width=\textwidth]{../illustrations/graphviz/carStateMachine_parking}
\end{center}

\noindent
Samtidig skal der ske noget, når vi skifter tilstand:
\begin{itemize}
\item Når vi skifter tilstand fra BACKWARD til PARKING, skal vi sætte
  en timer. Det gør vi ved at sætte en global variabel carParkingTimer
  til 600 (10 sekunder).
\item Når vi er i PARKING tilstanden, trækker vi en fra carParkingTimer
  i hver iteration af draw.
\end{itemize}
\end{exercisebox}

\begin{exercisebox}[adjusted title=Kun opladning når der er parkeret]
Den sidste udvidelse af tilstandsmaskinen gør at vi kun kan oplade
bilen imens vi er parkeret.

\hspace{-3cm}
\includegraphics[width=1.45\textwidth]{../illustrations/graphviz/carStateMachine_charging}

\noindent
Der tilføjes tre skift mellem tilstande:
\begin{itemize}
\item Tilstanden skifter frem og tilbage mellem PARKING og CHARGING,
  når der trykkes på tasten 'c'
\item Når man er i CHARGING tilstanden, skal bilen også begynde at
  køre igen når \ttpy{carParkingTimer} er 0.
\end{itemize}

\noindent
Samtidig skal der tilføjes kode så bilen lader automatisk, så længe
\ttpy{carState == CHARGING}.
\end{exercisebox}

\begin{exercisebox}[adjusted title=CO2 udledning]
Den sidste del af projektet er at få CO2-udledningen til at afhænge af
vindhastigheden. Vi opretter en global variabel \ttpy{totalEmission}
og sætter den til:
\begin{python}
if carState == "CHARGING":
    totalEmission = totalEmission + 4 - windSpeed
\end{python}

\noindent
Når windSpeed er 4 m/s, vil der derfor ikke forurenes, når windSpeed er
3 m/s, vil der forurenes en lille smule og så videre.
\end{exercisebox}

% \newpage
% \stepcounter{handout}
% \chapter{Objekter}
% Med objekter kan vi gruppere mange værdier sammen. Vi skal nu lave
% fiske-objekt der gemmer alle værdierne om en fisk (x, y, hastighed,
% øjenstørrelse).

% \begin{python}
% def makeFish(x, y, velocity, eyeSize):
%     return { "x" : x,
%              "y" : y,
%              "velocity" : velocity,
%              "eyeSize" : eyeSize }

% fish1 = makeFish(200, 50, 1, 15)
% print(fish1)
% print(fish1["x"])
% print(fish1["y"])
% print(fish1["velocity"])
% print(fish1["eyeSize"])
% \end{python}

% \begin{python}
% def drawFish(fish):
%     x = fish["x"]
%     y = fish["y"]
%     velocity = fish["velocity"]
%     eyeSize = fish["eyeSize"]
%     ... tegne kommandoer som tidligere ...

% def moveFish(fish):
%     fish["x"] = fish["x"] + 1
% \end{python}



% \newpage
% \stepcounter{handout}
% \chapter{Rotation i Processing.py}
% Hvis vi gerne vil rotere en tegning i Processing, så gøres det via et
% par kommandoer til at ændre hele koordinatsystemet.

% \chapter{Flyt origo, (0,0)}
% Med kommandoen \ttpy{translate(x,y)}, kan vi ændre hele
% koordinatsystemet, så vi kan tegne med origo et andet sted end øverst
% til venstre (med andre ord: flytte hvor vi har koordinat $(0,0)$).
% \begin{python}
% translate(150, 100)
% ellipse(0, 0, 120, 75)
% triangle(-60, 0, -90, -30, -90, 30)
% \end{python}

% \chapter{Roter koordinatsystemet}
% Rotation af koordinatsystemet gøres med kommandoen \ttpy{rotate(rad)} og
% det påvirker så alt der tegnes efterfølgende. Rotationen foregår rundt
% om origo, og derfor vil man oftest først bruge translate til at flytte
% hen til det punkt der skal roteres om, før man roterer.

% \noindent
% Eksempel:
% \begin{python}
% translate(150, 100)
% rotate(PI/2)
% ellipse(0, 0, 120, 75)
% triangle(-60, 0, -90, -30, -90, 30)
% \end{python}

% \noindent
% \textbf{Bemærk:} \ttpy{rotate} tager en værdi i radianer ($0-2\pi$)
% som argument. Husk at $\pi$ svarer til 180 grader, $\pi/4$ derfor til
% 45 grader, osv. Omregningsformlen fra matematik er:

% $$\textrm{radianer} = \pi \times \frac{\textrm{grader}}{180}$$

% \chapter{Genetabler det gamle koordinatsystem}
% Man kan selvfølgelig gå tilbage ved fx at rotere koordinatsystemet
% modsat, så man igen kan tegne noget der ikke er roteret. For eksempel
% \texttt{rotate(-PI/2)}. Der er dog en nemmere metode med kommandoerne
% \ttpy{pushMatrix()} og \ttpy{popMatrix()} (de tager ikke nogle
% argumenter).\\

% \begin{tabular}{lp{0.70\linewidth}}
%   \ttpy$pushMatrix()$ & Gem den nuværende tilstand af koordinatsystemet. \\
%   ~ & ~ \\
%   \ttpy$popMatrix()$ & Fortryd ændringer af koordinatsystemet, og genetabler den senest
%   gemte tilstand (fra før kaldet til \ttpy$pushMatrix()$) \\
% \end{tabular}

\end{document}
